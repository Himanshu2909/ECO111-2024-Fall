\documentclass[11pt]{article}
\usepackage[utf8]{inputenc}	% Para caracteres en español
\usepackage{amsmath,amsthm,amsfonts,amssymb,amscd}
\usepackage{multirow,booktabs}
\usepackage[table]{xcolor}
\usepackage{fullpage}
\usepackage{lastpage}
\usepackage{enumitem}
\usepackage{fancyhdr}
\usepackage{mathrsfs}
\usepackage{wrapfig}
\usepackage{setspace}
\usepackage{calc}
\usepackage{multicol}
\usepackage{cancel}
\usepackage[retainorgcmds]{IEEEtrantools}
\usepackage[margin=3cm]{geometry}
\usepackage{amsmath}
\newlength{\tabcont}
\setlength{\parindent}{0.0in}
\setlength{\parskip}{0.05in}
\usepackage{empheq}
\usepackage{framed}
\usepackage[most]{tcolorbox}
\usepackage{xcolor}
\usepackage[hidelinks]{hyperref}
\colorlet{shadecolor}{orange!15}
\parindent 0in
\parskip 12pt
\geometry{margin=1in, headsep=0.25in}
\theoremstyle{definition}
\newtheorem{defn}{Definition}
\newtheorem{reg}{Rule}
\newtheorem{exer}{Exercise}
\newtheorem{note}{Note}
\begin{document}

%Change this for headings
\setcounter{section}{0}
\title{Lecture 1 Class Notes}

\thispagestyle{empty}

\begin{center}
{\LARGE \bf Lecture 1 Class Notes}\\
{\large ECO111: Introduction to Econimcs}\\
Fall 2024
\end{center}
%Heading Ends


\tableofcontents


%Content Starts
\section{Topic of the Day}
\subsection{Gross Domestic Product}
\subsection{Inflation and Real GDP per capita}
Removing food categories in rich developed contries to calculate inflation rate might work as most people are able to afford basic food, and access to food is not a major issue for those countries. But in India, this removal is not practical as food sector encompasses as a large sector and not all people have access to basic food.

Compound Annual Gorwth Rate is a better metric


\subsection{Purchasing Power Parity}

\subsection{Inequalities}
Two types: Inequalities amongst people in the same nation ad between grwoth and groewth rates of different countries
Metrics to measure: Rich-Poor Ratio and 90-10 ratio

\subsection{Ratio Scale}


\end{document}
